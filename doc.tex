% ex: tw=74 
\documentclass[dvips,10pt,twocolumn]{article}
\usepackage[utf8]{inputenc}
\usepackage{amsmath}
\usepackage{xkeyval}
\usepackage[bookmarksnumbered,frenchlinks]{hyperref}
\hypersetup{pdfborder=0 0 0}
\usepackage{multirow}
\usepackage{todonotes}
\usepackage[english]{babel}
\usepackage{fullpage}
\usepackage{tabulary}
\usepackage{tabularx}
\usepackage{natbib}
\usepackage[all]{hypcap}
\title{802.16-2009 OFDM PHY baseband}
\author{Jonathan Huang \and Cody Schafer \and Luke Steepy}
\date{\today}
\begin{document}
\maketitle

\section{Project Overview}
We are developing a minimal 802.16-2009 OFDM Transmitter PHY.  This PHY
supports only QPSK-1/2 modulation-convolution rate. Additionally, any
components not relevant to unlicensed bands are omitted.  All optional items
are omitted.

This PHY requires the cooperation of a MAC layer implementation which is
compliant with the 802.16-2009 standard in order for the compliance
expectations (\autoref{sec:comply}) to be met.

	\subsection{End Market Expectations}
	Currently IEEE 802.16, also known as ``WiMAX'', has seen little
	deployment in the United States and western Europe while being
	moderately deployed in Asian nations and Eastern European markets.
	To be useful in western areas, WiMAX has to effectively compete
	with both consumer owned Wifi hot-spots and the various cell phone
	networks. Given the impressively low cost of both cellular and Wifi
	(802.11) systems combined with the expectation that they work well
	in mobile devices, this physical layer (PHY) targets both
	simplicity (to reduce cost) and reduced power consumption.

	\subsection{Degree of standards compliance and scope limitations}
	\label{sec:comply}
	The 802.16 standard specifies multiple layers of a WiMAX device,
	including a ``Service-specific convergence
	sublayer''~\cite[section 5]{IEEE:802.16}, a MAC
	sublayer~\cite[section 6]{IEEE:802.16}, a Security
	sublayer~\cite[section 7]{IEEE:802.16}, and the Physical
	layer~\cite[section 8]{IEEE:802.16}. 
	
	The PHY, while being described within section 8 of the 802.16-2009
	document, is additionally subject to certain constraints and
	requirements stipulated within the other sections.  These sections
	where only utilized to the extent to which they apply to design of
	the PHY component.

	Applicable sections of \cite{IEEE:802.16}.
	\begin{itemize}
		\item Section 8.3 - OFDM description
		\item Section 8.3.3.4.1 - Data Modulation: implementation
			is not compliant with this section, only QPSK
			modulation is supported.
		\item 8.3.3.4.3 - Rate ID encodings: not compliant with
			this section, only the QPSK-1/2 rate ID is supported.
		\item 8.3.5.1.1 - DL subchannelization: Rate ID
	\end{itemize}

\section{Electrical Characteristics}
All voltages of logic levels are defined by the FPGA on which the product's
code is placed.

\section{Frame Interface}
\label{sec:frame}
The MAC interfaces with the PHY module by sending frames (with the
appropriate headers and padding included) as a stream of bits.  These bits
are clocked via the \textbf{mac\_clk\_in} line, and must only be sent when
the \textbf{mac\_sending\_frame} line is high.  The state of the
\textbf{mac\_sending\_frame} line must be low when not sending a frame, and
must be lowered and raised between frames that would otherwise be abutting.
The frequency of \textbf{mac\_clk\_in} must be less than or equal to half
of the \textbf{phy\_clock}.

The \textbf{phy\_ready} line is a signal to the mac that it is ready for a
new frame to be inputed, and should not be ignored.

The structure of individual frames detailed in section 8.3.5 of IEEE
802.16-2009.  Each frame contains a header, up to 4 DL sub-frames (each with
their own structure also defined in IEEE802.16-2009 section 8.3.5) and some
number of UL sub-frames.  Note that due to fixing certain implementation
parameters, some fields are constrained further than mentioned within the
standard.

\begin{description}
	
	\item[Frame Header - Rate\_ID] is fixed at '1' indicating QPSK
		modulation with a code rate of $1 over 2$.
	
	\item[] \end{description}

\section{Burst Interface}
\label{sec:burst}

Bursts are composed of multiple OFDM symbols. OFDM symbols are composed of
multiple sub-channels(??).

\begin{table*}
	\begin{tabularx}{\textwidth}{c|c|c|X}
		\label{tbl:burst-io}
		Name & Width & Direction & Description\\ \hline

		\texttt{iuc} & 4 & I & The UIUC or DIUC field (depending on uplink
		or downlink state). Used by randomizer\\
		
		\texttt{bsid} & 4 & I & Used by randomizer \\
		
		\texttt{frame\_num} & 4 & I & Used by randomizer \\

	\end{tabularx}
	\caption{Per Burst interface to transmit hardware}
\end{table*}

	
\section{Internal Interfaces}
This section covers the interfaces for internal structures within the PHY
transmitter which are not exposed for use by the MAC or any other interfacing
hardware.

\autoref{tbl:internal-common-wire} shows some of the common signals for
the internal interfaces.

\begin{table*}
\begin{tabulary}{\textwidth}{c|C|L}
	\label{tbl:internal-common-wire}
	Name & Active Level & Meaning \\ \hline
	
	\textbf{*\_valid} & high & Indicate the source outputting the signal
	is also outputting data (clocked via the \textbf{phy\_clock}) which
	should be processed by the next item in the chain. \\

	\textbf{*\_bits} & high & A serial stream of bits clocked by
	\textbf{phy\_clock}. Only valid when the corresponding
	\textbf{*\_valid} line is also active. \\

	\textbf{*\_flag} & high & Set active for a single clock cycle
	before becoming inactive again.
\end{tabulary}
\caption{Common signals used internally}
\end{table*}

Each block of the transmitter connected via direct wiring (without a
buffer) is given the same clock. Each block reads its inputs on alternating
clock edges such that two adjacent units read and write on different edges.
This is done so that outputted data does not change while being read.

	\subsection{Header Decoder}
	\label{sec:header}
	The Randomizer (\autoref{sec:rand}) requires the locations of the
	PDUs (packet data units, 2 per frame) within each frame as well as
	the locations of 'Bursts' within each PDU to reinitialize itself at
	various points in the bit stream. As all of this data is contained
	in the PDU headers, this block determines the values of the needed
	fields and forwards them to the Control Logic block
	(\autoref{sec:ctrl}). 

	This block has the unique challenge of additionally reading it's
	bit-stream at the clock specified by the MAC,
	\textbf{mac\_bit\_clk\_in}.

	\begin{description}
		\item{Inputs:} \begin{description}
			\item[bits\_to\_phy], input bit-stream,  clocked via
				\textbf{mac\_bit\_clk\_in}.  No enable
				used. Uses \textbf{mac\_frame\_flag} to
				determine when a new frame is inbound.
			\item[mac\_bit\_clk\_in], as indicated above.
		\end{description}
		\item{Outputs:} \begin{description}
			\item[bits\_to\_rand] is the output bit-stream.
			\item[en\_to\_rand] indicates the output bit-stream
				is valid.
			\item[??] All of the data in some format, sent to
				the control logic so that it can reset the
				randomizer at the appropriate times and
				forward the subchannelization data to other
				blocks of logic.
		\end{description}
	\end{description}

	Estimated gates: 1k \\
	Responsible person: Cody

	\subsection{Randomizer}
	\label{sec:rand}
	The randomizer operates as a shift register with an initial value
	determined by the PDU (packet data unit, one half of a frame) header
	and whether it is a UL (uplink) or DL (downlink) PDU.

	\begin{description}
		\item{Inputs:}
		\begin{description}
			\item[bits\_to\_rand], the input bit-stream
			\item[en\_to\_rand], indicates the input bit-stream
				is valid.
			\item[reset\_rand], indicates a reload of the
				register is needed, only high for a single
				bit. Causes the reading of \textbf{rand\_iv}
				into an internal register.
			\item[rand\_iv{[15]}], a new vector to load into
				the shift register.
		\end{description}
		\item{Outputs:}
		\begin{description}
			\item[bits\_to\_fec], bit stream clocked via the
				main clock.
			\item[en\_to\_fec], indicates that the incoming
				bit-stream is valid.
		\end{description}
	\end{description}

	Estimated Gates: 1k \\
	Responsible Person: Cody.

	\subsection{Forward Error Correction}
	\label{sec:fec}
	This is a Reed-Solomon and convolution coding combination
	which is applied per frame. Within the standard, different
	RS (Reed-Solomon) codes and CC (convolution code) rates are
	used for varying modulation types. As we have fixed the
	modulation and code rate to QPSK-1/2, one two of the RS
	code and CC rate pairs are needed, as indicated in the
	\autoref{tbl:fec}.
	
	\begin{table*}
		\begin{tabular}{p{2cm}|p{2cm}|p{2cm}|p{2cm}|p{2cm}|p{2cm}}
		\label{tbl:fec}
			Modulation & Uncoded block size (bytes) &
			Coded block size (bytes) & Overall coding
			rate & RS code & CC code rate \\ \hline
			QPSK & 24 & 48 & 1/2 & (32,24,4) & 2/3 \\
		\end{tabular}
		\caption{Forward Error correction rates}
	\end{table*}

	\begin{description}
		\item{Inputs:} \begin{description}
			\item[bits\_to\_fec], from header decoder
				(\autoref{sec:header})
			\item[en\_to\_fec], indicates the incoming
				bit-stream is valid.
		\end{description}
		\item{Outputs:} \begin{description}
			\item[bits\_from\_fec], output bit-stream
			\item[en\_from\_fec], set high to indicate
				output bit-stream is valid.
		\end{description}
	\end{description}

	Estimated Gates: 2k \\
	Responsible Person: Luke.

	\subsection{FEC to Interleaving Circular buffer}
	\label{sec:fec_buffer}

	\begin{description}
		\item{Inputs:} \begin{description}
			\item[bits\_from\_fec]
			\item[en\_from\_fec], indicates the incoming
				bit-stream is valid
			\item[fec\_buf\_adv\_flag] set high for a
				single clock cycle by the consumer to
				indicate it has consumed the
				presently outputted data block.
		\end{description}
		\item{Outputs:} \begin{description}
			\item[fec\_buffer\_flag], 
			\item[fec\_buffer\_out{[96]}], set high to indicate
				output bit-stream is valid.
		\end{description}
	\end{description}

	Takes \textbf{bits\_from\_fec} (triggered via
	\textbf{en\_from\_fec}) and places them in a circular
	buffer. When the number of buffered bits equals the block
	size of the Interleaver (set at 96 bytes or 768 bits), the
	\textbf{fec\_buffer\_done} flag is set. Data is outputted
	via \textbf{fec\_buffer\_out[96]}

	Estimated Gates: 1K \\
	Responsible Person: Cody

	\subsection{Interleaving}
	\label{sec:interleaving}
	
	All encoded data bits will be interleaved by this block interleaver. The
	block size is defined by the number of coded bits per allocated subchannels 
	per OFDM symbol ($N_{cbps}$). 
 
	 \begin{center}
  	\begin{tabular}{|c|c|c|c|c|c|}
	\hline
	\multicolumn{6}{|c|}{Block sizes of the Interleaver} \\ \hline
	\multicolumn{6}{|c|}{$N_{cbps}$} \\ \hline
  	& 16 subchannels & 8 & 4 & 2 & 1 \\ \hline
	QPSK & 384 & 192 & 96 & 48 & 24 \\ \hline
	\end{tabular}
  	\end{center}

	Interleaving is a two step type operation. The first step ensures adjacent 
	coded bits are mapped onto nonadjacent subcarriers and the second step 
	ensures bits are mapped alternately onto less and more significant bits of
	the constellation. The second step is done to avoid long runs of lowly
	reliable bits.
	 
	The permutations are dependent on the number of coded bits per subcarrier
	($N_{cpc}$), which is defined by the type of modulation and the index of the
	coded bit before permutation (k).

	For our implementation, as we are only using QPSK modulation,
	$N_{cpc} = 2$.
	
	k : index of coded bit before first permutation

	$m_{k}$ : index after first permutation

	$j_{k}$ : index after second permutation 

	First permutation equation :
	\begin{equation}
	m_k = (N_{cbps}/12)*k_{mod12}+floor(k/12)
	\end{equation}

	Second permutation equation :
	\begin{equation}
	s=ceil(N_{cpc}/2) \\
	\end{equation}
	\begin{eqnarray}
	j_k = s*floor(m_{k}/s)+(m_{k}+N_{cbps}- \nonumber \\
		floor(12*m_k/N_{cbps}))_{mod(s)}
	\end{eqnarray}

	After the interleaving the first bit out of the interleaver would map to the
	MSB for the constellation mapping. 
	
	Input : 
	fec\_buffer\_out from Buffer, block of bits is read along this line
	fec\_buffer\_done from Buffer, block of bits is ready to read
	interleave\_reset from Control Logic
	
	Output :
	interleave\_out\_enable to Constellation Mapping, output line is ready to read
	interleaved\_bits to Constellation Mapping, output line

	\subsubsection{Testing and Verification}
	
	The Interleaver operation can be verified by inputting a known block of bits
	and observing the output. With the equations defined in the IEEE 802.16
	document, the appropriate interleaver output can be calculated and compared
	directly. 
 
	Estimated Gates: 2K \\
	Responsible Person: John

	\subsection{Constellation Mapping}
	\label{sec:constellation}

	Input is bit stream, output is (I,Q).
	
	Estimated Gates: 4K \\
	Responsible Person: Luke

	\subsection{Pilot Sub-carrier Insertion}
	\label{sec:pilot}

	Input is (I,Q) pair, output is (I,Q) pair with pilot subcarriers
	inserted at some points (fixed).

	Estimated Gates: 1K \\
	Responsible Person: Luke

	\subsection{Sub-carrier to IFFT Buffer}
	\label{sec:ifft-buffer}
	Insertion into this buffer is ordered but will have random
	jumps around different UL \& DL bursts.

	Block Size = blk\_siz = 200 (the total number of used
	subcarriers)
	
	Buffers blk\_size items each of which has width 2 bits.
	QPSK utilizes both I and Q, amplitude and phase, each with
	a granularity of 2. This means that each item (a pair of I
	\& Q) has 4 possible values and thus 2 bits are needed for
	each.

	Estimated Gates: 2k \\
	Responsible Person: Cody.

	\subsection{IFFT}
	\label{sec:ifft}
	IFFT Size: 256 items (each being a complex number)
	\begin{description}
		\item{Inputs:}
		\begin{description}
			\item[ifft\_buf\_has\_block] The count of the last
				block present in the Sub to IFFT buffer
				(\autoref{sec:ifft-buffer})
			
			\item[ifft\_buf\_data{[200][2][??ndepth]}] access
				to the data at the present 'tail' location
				in the buffer
		\end{description}
		
		\item{Outputs:}
		\begin{description}
			\item[ifft\_data\_ready] high when the IFFT has
				processed data.
		\end{description}
	\end{description}

	Estimated Gates: 10k.
	Responsible Person: John.

	\subsection{Cyclic Prefix}
	\label{sec:cyclic_prefix}
	\begin{description}
		\item{Inputs:}
		\begin{description}
			\item[ifft\_done] indicates the IFFT has
				data to be read out.
			\item[ifft\_data] : single bit of data from
				the IFFT.
		\end{description}
		\item{Outputs:}
		\begin{description}
			\item[ifft\_shift\_clk] clocks out the data from the IFFT.
			\item[bits\_cyclic\_appended{[2]}]: sequence of
				I\&Q with the prefix appended.
				This is sent directly to the ADC.
		\end{description}
	\end{description}

	Appends the last Tg items of the previous frame to the start of the
	present frame.  Reads 256 2 bit items from the IFFT, appending Tg
	items to the start of it. It also stores the last Tg items so that
	they may be appended to the next frame from the IFFT.

	Size is fixed at ??

	Estimated Gates: 2k.
	Responsible Person: John.

	\subsection{Control Logic}
	\label{sec:ctrl}

\section{Testing Considerations}
\section{Economic Analysis}
\section{Power Consumption}
\section{FPGA Implementation}
  \subsection{Sampling and Clock Rate}
    From the IEEE 802.16 standards document, the sampling rate for
    256-OFDM is defined as
    \begin{equation}
    F_s = BW * 7/6
    \end{equation}
  
  \begin{center}
  \begin{tabular}{c|c}
  Channel Bandwidth & Sampling Rate \\ \hline
  14 MHz & 16.333 MHz \\
  7 MHz & 8.166 MHz \\
  3.5 MHz & 4.083 MHz \\
  1.75 MHz & 2.042 MHz
  \end{tabular}
  \end{center}
  
  This design should be able to run on both the Spartan-3 FPGA and Virtex 4 LX60
  FPGA as they both supply a 500 MHz clock.
  
  \subsection{Power Calculation}
  
  Calculate power based on required clock frequency for the device as well as the
  FPGA specifications for voltage inputs.
  
  Absolute Maximum rating for $V_{in}$ : 4.4 V
  
  Clock frequency required for a 3.5 MHz channel bandwidth : 4.083 MHz
  
  Input capacitance over recommended operating conditions : 10 pF
  \begin{center}
  \begin{equation}
  P = C * V^2 * f = 10 pF * (4.4 V)^2 * 4.083 MHz = 0.18 mW
  \end{equation}
  \end{center}
  
	%\subsection{Pin outs}
	%XC3S200-FT256
	%//Top and Bottom View Package FT256 -   %http://www.xilinx.com/support/documentation/package_specs/ft256.pdf 
	%\section{Datasheet(Spartan-3 FPGA}
	%\subsection{Electrical Characteristics}
	%\begin{tabular}
	%\end{tabular}

\bibliography{doc}{}
\bibliographystyle{ieeetr}
\end{document}
