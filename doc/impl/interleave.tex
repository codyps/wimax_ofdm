
\subsection{Interleaver}
\label{sec:interleaver}
\begin{description}
	\item[Estimated gates] 2000
	\item[Estimated data bitstream delay] About 392 clocks.
\end{description}
	
\begin{table*}
	\begin{tabularx}{\linewidth}{c|c|c|X}
		\label{tbl:interleaver-io}
		Name & Width & Direction & Description \\ \hline
		\wire{reset} & 1 & I & resets the device state. \\
		\wire{clk}   & 1 & I & clocking. \\
		\wire{in\_blk} & 384 & I & Block of input data. \\
		\wire{in\_blk\_valid} & 1 & I & Indicates a block of input data on \wire{in\_blk} is valid and should be processed. \\
		\wire{rate\_id} & 3 & I & The modulation type. Possible values can be found in \autoref{tbl:rate-id}. \\
		\wire{subchan\_ct} & 3 & I & Indicates whether the number of subchannels is 1, 2, 4, 8, or 16. These are paired with values 0, 1, 2, 3, and 4, 
			respectively. The result of other values is undefined. This input allows determintion on how many bits in \wire{in\_blk} should be processed. \\
		\wire{out\_blk} & 384 & O & Interleaved block which is then outputted. \\
		\wire{out\_blk\_valid} & 1 & O & When high, indicates that \wire{out\_blk} contains valid data which should be processed by the next item in the pipeline. 
\end{tabularx} \caption{Interleaver I/O description} \end{table*}

\begin{table*}
	\begin{tabularx}{\linewidth}{X|X|X|X|X|X}
		Subchannels & 16 & 8 & 4 & 2 & 1 \\ \hline
		BPSK   & 192  & 96  & 48  & 24  & 12 \\
		QPSK   & 384  & 192 & 96  & 48  & 24 \\
		16-QAM & 768  & 384 & 192 & 96  & 48 \\
		64-QAM & 1152 & 576 & 288 & 144 & 72
	\end{tabularx} 
	\caption{Values of $N_{cbps}$. Note
	that other than the values listed for QPSK, none of these items are
	supported by this device.}
	\label{tbl:Ncbps}
\end{table*}

All encoded data bits will be interleaved by this block interleaver. The
block size is defined by the number of coded bits per allocated subchannels 
per OFDM symbol ($N_{cbps}$). $N_{cbps}$ is determined by a
combination of the number of subchannels and the modulation used, see
\autoref{tbl:Ncbps} for a listing of possible values.

Interleaving is a two step type operation. The first step ensures adjacent 
coded bits are mapped onto nonadjacent subcarriers and the second step 
ensures bits are mapped alternately onto less and more significant bits of
the constellation. The second step is done to avoid long runs of lowly
reliable bits.
 
The permutations are dependent on the number of coded bits per subcarrier
($N_{cpc}$), which is defined by the type of modulation and the index of the
coded bit before permutation (k).

For our implementation, as we are only using QPSK modulation,
$N_{cpc} = 2$.

k : index of coded bit before first permutation

$m_{k}$ : index after first permutation

$j_{k}$ : index after second permutation 

First permutation equation :
\begin{equation}
m_k = (N_{cbps}/12)*k_{mod12}+floor(k/12)
\end{equation}

Second permutation equation :
\begin{equation}
s=ceil(N_{cpc}/2)
\end{equation}

\begin{eqnarray}
j_k = s*floor(m_{k}/s)+(m_{k}+N_{cbps}- \nonumber \\
	floor(12*m_k/N_{cbps}))_{mod(s)}
\end{eqnarray}

After the interleaving the first bit out of the interleaver would map to the
MSB for the constellation mapping. 

	\subsubsection{Testing and Verification}
	The Interleaver operation can be verified by inputting a known block
	of bits and observing the output. The standard document provides a few of these for our use.

	The verilog code for this module was generated via a python script which directly impliments the permutation equations (\autoref{lst:interleave-gen}). Additionally, a partial non-generated implimentation was created, the equations for which are verified by the same script which generates interleaver code.

	\subsubsection{Implimentation Notes}

Due to the complexity of the permutation, code was written which generates the verilog implimentation for all posible variations in $N_{cpc}$. In a further iteration of the transmitter, a multiplexer would be used to switch to the appropriate interleaver for each type of modulation. A partial implimentation of this multiplexing was written.

The interleaving operation is such that bits much later in the bitstream a swaped with those much earlier in the bitstream. Due to this, the entire OFDM symbol ($N_{cbps}$ bits) must be read into the module prior to a single 1 clock operation to swap the bytes as required. This leads to a rather high cost in delay for this module.