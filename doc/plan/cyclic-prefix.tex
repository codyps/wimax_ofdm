% CYCLIC PRERIX


\subsection{Cyclic Prefix}
\label{sec:cyclic_prefix}
\begin{description}
	\item[Estimated gates] 2000
	\item[Estimated data bitstream delay] About 1050 clocks.
\end{description}

The Cyclic Prefix is used to act as protection from Intersymbol Interference 
by duplicating a certain number of symbols defined by the $CP_rate$.

The Cyclic Prefix appends to the front of the symbol a copy of the last
$256/CP_{rate}$ ($N_{IFFT}/CP_{rate}$) bits. OFDM PHY should allow for four
(4) rates as defined by 2 bits in the PHY Mode ID field. 

It reads 256 2 bit items from the IFFT into 2 256 blocks of memory.
On output, the appropriate number of bits are read from the end of 
memory by offset.

At a rate of $1/4$, 64 bits would need to be read from the end and 
at a rate of $1/32$, 8 bits would need to be read.

The sample size would then increase from 256 bits to anywhere between
264 to 320 bits.

For our implementation, we will set the CP rate at $1/16$ which leads to
a sample size of 272 bits.

This process would take about about 512 clocks to read in the data to fill
memory and about 528 clocks to write out. 

The total time would include clocks to set and check valid lines and would
total to about 1050 clocks. 

\begin{table}
	\begin{tabulary}{\linewidth}{C|C}
		Value of \wire{param\_G} & Ratio of CP time to ``useful'' time. \\ \hline

		0b00 & 1/4 \\
		0b01 & 1/8 \\
		0b10 & 1/16 \\
		0b11 & 1/32

	\end{tabulary}
	\caption{Values of OFDM parameter G as presented on \wire{param\_G}}
	\label{tbl:param-g-vals}

\end{table}

\begin{table*}
	\begin{tabularx}{\linewidth}{c|c|c|X}
		Name & Direction & Width & Description \\ \hline

		\wire{param\_G} & I & 2 & The meanings of valid values
		are listed in \autoref{tbl:param-g-vals}. Indicates the
		fraction of CP time to ``useful'' time.
		\\

		\wire{cp\_in\_bits} & I & 1 & Input bitstream. \\
	
		\wire{cp\_in\_valid} & I & 1 & Indicates the input
		bitstream \wire{cp\_in\_bits} is valid. \\

		\wire{cp\_out\_bits} & O & 1 & Output bitstream. \\
	
		\wire{cp\_out\_valid} & O & 1 & Indicates the output
		bitstream \wire{cp\_out\_bits} is valid.
	\end{tabularx}
	\caption{Cyclic Prefix Inputs and Outputs}
	\label{tbl:cp-io}	
\end{table*}

\autoref{tbl:cp-io} lists the inputs and outputs to the cyclic
prefix unit.

Appends the last Tg items of the previous frame to the start of the
present frame.  Reads 256 by 2 16 bit items from the IFFT, appending Tg
items to the start of it. It also stores the last Tg items so that
they may be appended to the next frame from the IFFT.


\comment{\todo[inline]{Maximum Size needed?}}
