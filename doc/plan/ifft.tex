% IFFT

\subsection{IFFT}
\label{sec:ifft}
\begin{description}
	\item[Estimated gates] 10,000
	\item[Estimated data bitstream delay] 
\end{description}

The IFFT must process 256 complex numbers with a depth of 16 bits for each
component (16 bits for the Real component and an another 16 bits for the
imaginary).

See \autoref{tbl:ifft-io} for a description of the inputs and outputs.
	
	Buffers blk\_size items each of which has width 2 bits.
	QPSK utilizes both I and Q, amplitude and phase, each with
	a granularity of 2. This means that each item (a pair of I
	\& Q) has 4 possible values and thus 2 bits are needed for
	each.

	The IFFT takes in 256-bit complex samples and outputs to the Cyclic Prefix.

\begin{table*}
	\begin{tabularx}{\linewidth}{c|c|c|X}
		\label{tbl:ifft-io}
			
	Name & Width & Direction & Description \\ \hline

	\wire{reset} & 1 & I & Resets the device state. \\

	\wire{clk}   & 1 & I & Clocking. \\

	\wire{in\_blk\_valid} & 1 & I & When high, indicates that the input block is
	valid and should be processed. \\

	\wire{in\_blk} & $200 \cdot 2 \cdot 16$ & I & Input data for the IFFT. Width of
	numbers is 16 bits, processing both Q and I. \\

	\wire{out\_blk} & $256 \cdot 2 \cdot 16$ & O & Output data after IFFT is preformed. Width of
	numbers is 16 bits, processing both Q and I. \\

	\wire{out\_blk\_valid} & 1 & O & Indicates the output block
	\wire{out\_blk} is valid and should be immediately copied (it will be
	replaced by the next block on the next clock cycle).

\end{tabularx}
\caption{IFFT input/output description.}
\end{table*}