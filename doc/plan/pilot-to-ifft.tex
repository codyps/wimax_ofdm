% IFFT INPUT BUFFER.

\subsection{Sub-carrier to IFFT Buffer}
\label{sec:ifft-buffer}
\begin{description}
	\item[Estimated gates] 2000
	\item[Estimated data bitstream delay] 1
\end{description}
Insertion into this buffer is ordered but will have random
jumps around different UL \& DL bursts.

Block Size = blk\_siz = 200 (the total number of used
subcarriers)

Buffers blk\_size items each of which has width 2 bits.
QPSK utilizes both I and Q, amplitude and phase, each with
a granularity of 2. This means that each item (a pair of I
\& Q) has 4 possible values and thus 2 bits are needed for
each.

%\todo[inline]{Pilot insertion input needs to be defined}
\begin{table*}\begin{tabularx}{\linewidth}{c|c|c|X}
	\label{tbl:ifft-buffer-io}
			
	Name & Width & Direction & Description \\ \hline

	\wire{reset} & 1 & I & Resets the device state. \\

	\wire{clk}   & 1 & I & Clocking. \\

	\wire{in\_blk} & $200 \cdot 2 \cdot 16$ & I & Input data from the Pilot Insertion. \\

	\wire{in\_blk\_valid} & 1 & I & When high, indicates that the input block is
	valid and should be processed. \\

	\wire{out\_blk} & $200 \cdot 2 \cdot 16$ & O & Output data to the IFFT. Width of
	numbers is 16 bits, processing both Q and I. \\

	\wire{out\_blk\_valid} & 1 & O & Indicates the output block
	\wire{out\_blk} is valid and should be imediately copied (it will be
	replaced by the next block on the next clock cycle).

\end{tabularx}
\caption{Sub-carrier to IFFT Buffer input/output description.}
\end{table*}

