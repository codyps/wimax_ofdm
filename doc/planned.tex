
%% FRAME

\subsection{Frame Handler}
\begin{description}
	\item[Estimated gates] 2000
	\item[Estimated bitstream delay] 2 clock cycles for a bit to be
		processed 
\end{description}

Is given a bitstream directly from the MAC device, expected to be able to
clock in this bit stream at the rate defined by the MAC. Data and
parsed configuration is sent to the Burst Handler (\autoref{sec:burst})
to be processed and outputed.

Inputs and outputs are listed in \autoref{tbl:frame-io}.

\begin{table} \begin{center} \begin{tabularx}{\linewidth}{c|c}
	\label{tbl:rate-id}
	\wire{rate\_id[3]} & Modulation RS-CC rate \\ \hline
	0 & BPSK-1/2 \\
	1 & QPSK-1/2 \\
	2 & QPSK-3/4 \\
	3 & 16-QAM-1/2 \\
	4 & 16-QAM-3/4 \\
	5 & 64-QAM-2/3 \\
	6 & 64-QAM-3/4 \\
	7 & Reserved
\end{tabularx} \caption{Possible values for \wire{rate\_id}. Note that
All values but 1 (QPSK-1/2) are unsupported by this device.}
\end{center} \end{table}

\begin{table*}
	\label{tbl:subchan}
	
	\caption{Possible values for \wire{subchan\_data}.}
\end{table*}

\begin{table*} \begin{tabularx}{\linewidth}{c|c|c|X} \label{tbl:frame-io}
	Name & Width & Direction & Description \\ \hline

	\wire{clk} & 1 & I & Clocks the logic block.\\

	\wire{reset} & 1 & I & Active low reset to initial state. \\


	\wire{frame\_in\_bits} & 1 & I & Input bitstream, clocked via
	\wire{frame\_in\_clk}. \\

	\wire{frame\_in\_clk} & 1 & I & Positive edge indicates that data on
	\wire{frame\_in\_bits} should be read. \\
	
	\wire{frame\_in\_valid} & 1 & I & Set high while the current series
	of bits is part of the same frame. \\

	\wire{burst\_reset} & 1 & O & reset line to the burst handler
	(\autoref{sec:burst}). \\

	\wire{burst\_bits} & 1 & O & Bits to the burst handler. \\

	\wire{burst\_valid} & 1 & O & indicates the bitstream
	\wire{burst\_bits} is valid and should be processed. \\


	\wire{frame\_num} & 4 & O & The frame number parsed from the header
	structures. \\

	\wire{iuc} & 4 & O & The UIUC or DIUC field, parsed from the header
	stuctures. Only valid for the currently outputed burst. \\

	\wire{subchan\_data} & 6 & O & See \autoref{tbl:frame-io}.
	\autoref{tbl:subchan} describes valid values. Per burst. \\

	\wire{rate\_id} & 3 & O & See \autoref{tbl:rate-id}. Per burst. \\

\end{tabularx} \caption{Frame Handler interface description} \end{table*}

%% BURST

\subsection{Burst Handler} \label{sec:burst}
\begin{description}
	\item[Estimated gates] 2000 
	\item[Estimated data bitstream delay] 2 clock cycles for a bit to be
		processed, could be cut to zero by bypassing.
\end{description}

Bursts are composed of multiple OFDM symbols. OFDM symbols are composed of
multiple sub-channels.\autoref{tbl:burst-io} shows the burst interface inputs and outputs.
Across a single burst initialization of the pipeline hardware will not
change.

\begin{table*} \begin{tabularx}{\linewidth}{c|c|c|X}
	\label{tbl:burst-io}
	Name & Width & Direction & Description \\ \hline
	
	\wire{reset} & 1 & I & Imediately resets the burst handler. Active
	low. \\

	\wire{clk} & 1 & I & Clocking. \\

	\wire{iuc} & 4 & I & The UIUC or DIUC field (depending on uplink
	or downlink state). Used by randomizer. \\
	
	\wire{bsid} & 4 & I & Used by randomizer. \\
	
	\wire{frame\_num} & 4 & I & Used by randomizer. \\

	\wire{burst\_in\_bits} & 1 & I & Input bitstream. \\

	\wire{burst\_in\_valid} & 1 &  I & Indicates the input bitstream
	\wire{burst\_in\_bits} is valid. \\

	\wire{rand\_iv} & 15 & O & The initialization for
	the randomizer's internal register. \\

	\wire{rand\_reload} & 1 & O & Instructs the randomizer to reload
	it's internal register with a new one outputed on
	\wire{rand\_iv}. \\

	\wire{burst\_out\_bits} & 1 & O & Output bitstream. Sent to the
	randomizer for first portion of processing. \\

	\wire{burst\_out\_valid} & 1 & O & Indicates output is valid.
	Also attached to the randomizer. \\

	\wire{subchan\_data\_in} & 6 & I & See \autoref{tbl:frame-io}.
	\autoref{tbl:subchan} describes valid values. \\

	\wire{subchan\_data\_out} & 6 & O & Output pair of
	\wire{subchan\_data\_in}, simple pass through. \\

	\wire{subchan\_ct} & 4 & O & Indicates the number of subchannels in
	use. Calculated from \wire{subchan\_data\_in}. \\

	\wire{rate\_id\_in} & 3 & I  & See \autoref{tbl:rate-id}. \\

	\wire{rate\_id\_out} & 3 & O & Pass-through pair of
	\wire{rate\_id\_in}. \\

\end{tabularx}
\caption{Interface definition for the Burst Handler.}
\end{table*}


